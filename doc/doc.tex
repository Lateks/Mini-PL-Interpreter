\documentclass[a4paper,11pt]{article}
\usepackage[top = 1.5 in, bottom = 1 in, left = 1.5 in, right = 1.5 in]{geometry}
\usepackage[utf8]{inputenc}
\usepackage[T1]{fontenc}
\usepackage{csquotes}
\usepackage[british]{babel}
\usepackage{lmodern}
\usepackage{url}
\usepackage{color}
\usepackage{graphicx}
\usepackage{setspace}
\usepackage{pdfpages}

\begin{document}
\title{Course project: Mini-PL interpreter}
\author{Laura Leppänen \\ Student number: 013302782 \\ Compilers, Spring 2012}
\date{\today}
\maketitle
\thispagestyle{empty}

\tableofcontents
\onehalfspacing

\newpage
\setcounter{page}{1}

\section{Interpreter implementation}

This section covers the general architecture of the interpreter, testing, building and running instuctions etc.

\subsection{Architecture}

\subsection{Error handling}

\subsection{Testing}

\subsection{Building and running the interpreter}

\section{The Mini-PL language}

This section covers issues related to scanning and parsing Mini-PL: token patterns, the context-free grammar, and the abstract syntax tree.

\subsection{Mini-PL token patterns}

\subsection{Context-free grammar}

\subsection{Abstract syntax trees}

\appendix
\section{Appendices}

Starting on the next page.

\includepdf[pages={1, 2}]{project_spec.pdf}

\includepdf[pages={1, 2}]{minipl_spec.pdf}

\end{document}
